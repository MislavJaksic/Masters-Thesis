\documentclass[times, utf8, diplomski]{fer}
\usepackage{booktabs}

\begin{document}

\thesisnumber{1954}

\title{Stvarnovremensko praćenje parametara ispravnosti rada u sustavu za raspodijeljenu obradu tokova podataka}

\author{Mislav Jakšić}

\maketitle

\izvornik

\zahvala{Hvala svima!}

\tableofcontents

\chapter{Uvod}

Raspodijeljeni sustavi su izrazito nepouzdani. Ispad bilo kojeg dijela sustava može izazvati pogreška u sklopovlju, operacijskom sustavu, programu ili mreži. Jedan ispad u raspodijeljenom sustavu lako može ulančati ispade. Ako se ispad ne proširi, sustav će i dalje patiti jer će program u stanju ispada i dalje uzimati računalno vrijeme i memoriju, a s njima neće obavljati koristan posao. U najgorem slučaju ispad može izazvati potpuno zatajenje sustava gdje je jedini lijek iznova pokrenuti sve njegove dijelove. U najboljem slučaju ispad će samo smanjiti učinkovitost sustava. Bez temeljitog nadzora raspodijeljenog sustava teško je otkriti da se dogodio ispad, a još teze otkloniti izvor problema.

Zadatak nadzora je otkriti ispad i njegov uzrok. !RASSUS-KNJIGA dijeli ispade na ispad procesa, pogreške u komunikaciji, vremenske pogreške, pogrešan odgovor i bizantske pogreške. Ako se ispad želi otkriti potrebno je pratiti vrijednosti koje ukazuju na ispad. Korisne vrijednosti mogu biti zauzeće memorije, brzina obrade zahtjeva, sadržaj poruke ili duljina uspostave komunikacijskog kanala. Nadzirani program vrijednosti predaje nadzorniku koji je čovjek ili program. Ako nadzor obavlja program on treba biti pouzdaniji od nadziranog programa inače je problem nadzora programa udvostručen umjesto riješen.

Prvi korak u izradi pouzdanog sustava nadzora je izrada pouzdanog sakupljača vrijednosti. Prije same izrade istražene su ideje, sukobljene su arhitekture i pregledana postojeća rješenja. Kako bi naglasak na područjima bio podjednak napravljen je sakupljač za izabrani raspodijeljeni sustav. Da je napravljen svestran sakupljač koji može sakupljati vrijednosti raznolikog sklopovlja, operacijskih sustava i programa misli o arhitekturi bile bi izražene na uštrp onima o izradi programa. Nadzirani sustav je Apache Kafka, popularni sustav za raspodijeljenu obradu tokova podataka.

\chapter{Sustavi za raspodijeljenu obradu tokova podataka}

\chapter{Postojeca rjesenja}

\chapter{Arhitektura rjesenja}

\chapter{Dizajn rjesenja}

\chapter{Izvedba rjesenja}

\chapter{Rezultati}

\chapter{Zaključak}
Zaključak.

\bibliography{literatura}
\bibliographystyle{fer}

\begin{sazetak}
Sažetak na hrvatskom jeziku.

\kljucnerijeci{Ključne riječi, odvojene zarezima.}
\end{sazetak}

\engtitle{Real-Time Health Monitoring in Distributed Data Stream Processing System}
\begin{abstract}
Abstract.

\keywords{Keywords.}
\end{abstract}

\end{document}
